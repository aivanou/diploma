\section*{Введение}
\addcontentsline{toc}{section}{Введение}

Целью данного дипломного проекта является разработка алгоритмов машинного обучения, которые позволяют улучшить выдачу поисковых систем. Кроме этого проект демонстрирует как можно использовать неявные пользовательские отзывы(логи сессий) для улучшения ранжирования. 

В последнее время мы наблюдаем бурное развитие интернета. Количество интернет порталов и сайтов растет с очень большой скоростью и сейчас практически невозможно найти нужную информацию без поисковых систем. В свою очередь, поисковые системы должны отфильтровать и отсортировать огромный объём данных, чтобы предоставить желаемую информацию. Данная задача называется задачей ранжирования. 

Обучение ранжированию — это класс задач машинного обучения с учителем, заключающихся в автоматическом подборе ранжирующей модели по обучающей выборке, состоящей из множества списков и заданных частичных порядков на элементах внутри каждого списка. Частичный порядок обычно задаётся путём указания оценки для каждого элемента (например, «релевантен» или «не релевантен»; возможно использование и более, чем двух градаций). Цель ранжирующей модели — наилучшим образом (в некотором смысле) приблизить и обобщить способ ранжирования в обучающей выборке на новые данные. Это ещё довольно молодая, бурно развивающаяся область исследований, возникшая в 2000-е годы с появлением интереса в области информационного поиска к применению методов машинного обучения к задачам ранжирования.

Для любого алгоритма машинного обучения требуется обучающая выборка. Основной единицей обучения является пара документ-запрос и релевантность документа, данному запросу. Но, в зависимости от алгоритма обучения используются три подхода: 
\begin{itemize}

	\item Поточечный подход. В поточечном подходе предполагается, что каждой паре запрос-документ поставлена в соответствие численная оценка. Задача обучения ранжированию сводится к построению регрессии: для каждой отдельной пары запрос-документ необходимо предсказать её оценку.

	\item Попарный подход. В попарном подходе обучение ранжированию сводится к построению бинарного классификатора, которому на вход поступают два документа, соответствующих одному и тому же запросу, и требуется определить, какой из них лучше.

	\item Списочный подход. Списочный подход заключается в построении модели, на вход которой поступают сразу все документы, соответствующие запросу, а на выходе получается их перестановка. Подгонка параметров модели осуществляется для прямой максимизации одной из перечисленных выше метрик ранжирования.

\end{itemize}

Для построения модели ранжирования, был выбран поточечный подход. В данном дипломном проекте модель ранжирования представляет собой градиентный бустинг на основе решающих деревьев. Для получения релевантности документов из поисковых сессий была применена динамическая байесова сеть