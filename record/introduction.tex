\sectioncentered*{Введение}
\addcontentsline{toc}{section}{Введение}

Целью данного дипломного проекта является разработка алгоритмов машинного обучения, которые позволяют улучшить выдачу поисковых систем. Кроме этого проект демонстрирует как можно использовать неявные пользовательские отзывы(логи сессий) для улучшения ранжирования. 

Поиск информации представляет собой процесс выявления в некотором множестве документов (текстов) всех тех, которые посвящены указанной теме (предмету), удовлетворяют заранее определенному условию поиска (запросу) или содержат необходимые (соответствующие информационной потребности) факты, сведения, данные.

Процесс поиска включает последовательность операций, направленных на сбор, обработку и предоставление информации.

В общем случае поиск информации состоит из четырех этапов:

\begin{itemize}
	\item определение (уточнение) информационной потребности и формулировка информационного запроса
	\item определение совокупности возможных держателей информационных массивов (источников)
	\item извлечение информации из выявленных информационных массивов;
	\item ознакомление с полученной информацией и оценка результатов поиска
\end{itemize}

Информационный поиск — большая междисциплинарная область науки, стоящая на пересечении когнитивной психологии, информатики, информационного дизайна, лингвистики, семиотики, и библиотечного дела.

Поиск информации — процесс выявления в массиве информации записей, удовлетворяющих заранее определенному условию поиска или запросу.

Информационный поиск рассматривает поиск информации в документах, поиск самих документов, извлечение метаданных из документов, поиск текста, изображений, видео и звука в локальных реляционных базах данных, в гипертекстовых базах данных таких, как Интернет и локальные интранет-системы.

Существует некоторая путаница, связанная с понятиями поиска данных, поиска документов, информационного поиска и текстового поиска. Тем не менее, каждое из этих направлений исследования обладает собственными методиками, практическими наработками и литературой.

В настоящее время информационный поиск — это бурно развивающаяся область науки, популярность которой обусловлено экспоненциальным ростом объемов информации, в частности в сети Интернет. Информационному поиску посвящена обширная литература и множество конференций.

В последнее время наблюдается бурное развитие интернета. Количество интернет порталов и сайтов растет с очень большой скоростью и сейчас практически невозможно найти нужную информацию без поисковых систем. В свою очередь, поисковые системы должны отфильтровать и отсортировать огромный объём данных, чтобы предоставить желаемую информацию. Данная задача называется задачей ранжирования. 

Обучение ранжированию — это класс задач машинного обучения с учителем, заключающихся в автоматическом подборе ранжирующей модели по обучающей выборке, состоящей из множества списков и заданных частичных порядков на элементах внутри каждого списка. Частичный порядок обычно задаётся путём указания оценки для каждого элемента (например, «релевантен» или «не релевантен»; возможно использование и более, чем двух градаций). Цель ранжирующей модели — наилучшим образом (в некотором смысле) приблизить и обобщить способ ранжирования в обучающей выборке на новые данные. Это ещё довольно молодая, бурно развивающаяся область исследований, возникшая в 2000-е годы с появлением интереса в области информационного поиска к применению методов машинного обучения к задачам ранжирования.

Для любого алгоритма машинного обучения требуется обучающая выборка. Основной единицей обучения является пара документ-запрос и релевантность документа, данному запросу. Но, в зависимости от алгоритма обучения используются три подхода: 

\begin{itemize}

	\item Поточечный подход. В поточечном подходе предполагается, что каждой паре запрос-документ поставлена в соответствие численная оценка. Задача обучения ранжированию сводится к построению регрессии: для каждой отдельной пары запрос-документ необходимо предсказать её оценку.

	\item Попарный подход. В попарном подходе обучение ранжированию сводится к построению бинарного классификатора, которому на вход поступают два документа, соответствующих одному и тому же запросу, и требуется определить, какой из них лучше.

	\item Списочный подход. Списочный подход заключается в построении модели, на вход которой поступают сразу все документы, соответствующие запросу, а на выходе получается их перестановка. Подгонка параметров модели осуществляется для прямой максимизации одной из перечисленных выше метрик ранжирования.

\end{itemize}

Для построения модели ранжирования, был выбран поточечный подход. В данном дипломном проекте модель ранжирования представляет собой градиентный бустинг на основе решающих деревьев. Для получения релевантности документов из поисковых сессий была применена динамическая байесова сеть

Задача обучения ранжированию на онове кликов нетривиальна. В ~\cite{joachims} было показано, что даже если документы в выдаче отсортированы в обратном порядке(от худшего к лучшему), пользователи будут сначало производить клики по документам, которые занимают первые места. Поэтому, недостаточно просто отсортировать все документы по количеству кликов. В всязи с этим разрабатываются спеуиальные математические модели, которые потом тренируются на обучающей выборке для нахождения релевантности документов.

\newpage