\section{Заключение}

Данный дипломный проект демонстрирует построение алгоритма ранжирования использую современные технологии. Была продемонстрирована возможность использования действий пользователей для улучшения ранжирования. Также, централльный алгоритм обучения построен на основе деревьев, что позволяет построить схематический вид и интерпретировать результаты в вид, понятный для людей.
 
В ходе разработки на практике были исследованы возможности построения надежного алгоритма ранжирования на основе бустинга.

Разработанная система полностью удовлетворяет требованиям, сформуливанным в исходных данных к дипломному проекту, и обладает рядом положительных характеристик:
\begin{itemize}
  \item общность, то есть возможность применения данного программного приложения в широком спектре информационного поиска, что существенно отличает задействованный метод от других современных методов,
  \item относительная простота как реализации, так и математической формулировки,
  \item устойчивость к зашумленности данных, благодаря информативности признаков
\end{itemize}

Существуют возможности усовершенствования проекта с целью повышения точности и надежности ранжирования результатов. Одним из способов является использование совокупности алгоритмов.

В работе было проведено технико-экономическое обоснование разработки системы. Произведенные расчеты показали, что разработка программного обеспечения является рентабельной. Программное обеспечение имеет короткий срок окупаемости.

\newpage