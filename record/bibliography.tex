\begin{thebibliography}{99}

\bibitem{joachims}
   Джочим, Т.
  \emph{Оптимизация поисковых систем с использованием кликов данных}./ 
  Т. Джочим // Конференция поисковых технологий --- 2009 -- С. 14-19.

\bibitem{cart_estimation_book}
 Бургес К.
  \emph{Использование бустинга в задаче ранжирования} / Пинг Ли, Кристофер Бургес - М. Вильямс - 2007 г.

\bibitem{cart_optimal3}
  Брамер М.
  \emph{Принципы дата майнинга}. / Макс Брамер --- М. Уолтерс ---  2007 г.

\bibitem{cart_optimal4}
  Горвард Т.
  \emph{Логика. Индуктивное программировани}. / Томас Горвард --- М. Открытые системы - 2003 г.

\bibitem{cart_optimal5}
  Денг Р., Рудгер П.
  \emph{Обобщение важности мер на многозначные атрибуты и их решения}. / Р. Денг, П. Рудгер  М. Физматкнига - 2011 г. 

\bibitem{ctr_improving1}
  Торстен Д. 
  \emph{Оптимизация поисковой программы используя клики данных}. /
  Д. Торстен // 2004 г.

\bibitem{ctr_improving2}
  Эрик Брилл, Сьюзен Димаис
  \emph{Улучшение ранжирования информационного поиска используя поведения пользователей}.
  2006 г.

\bibitem{martin}
  Мартин Дж. К.,
  \emph{Точная вероятностная метрика для расщепления деревьев регения}.
  1997 год. Стр. 257-291.

\bibitem{m_chain}
  Джуди Пирл,
  \emph{Причинность: модели, мышление и вывода}, 2-e изд.
  М.: Издательский дом ``Вильямс'',
  2009.

\bibitem{auc_book}
  Чарльз Линг, Джин Хуанг
  \emph{AUC: Статистически полная и более надежная мера оценки, чем точность измерения}.
  2003 г.

\bibitem{cart_book}
  Data Mining Conferention
  \emph{CART: classification and regression trees}.
  2006 г.

\bibitem{cart_optimal1}
  Хоуфали, Лоурент
  \emph{Построение оптимальных бинарных решающих деревьев является NP полной задачей}.
  2003., стр 15-17.

\bibitem{cart_optimal2}
  Мурфи С.
  \emph{Автоматическое создание решающих деревьев}.
  1998 г.

\bibitem{ccm}
  Том Минка, Майкл Тейлор
  \emph{Последовательная модель кликов}.
  2009 г. 

\bibitem{dbn}
  Оливер Чапеллер, Ya Zhang
  \emph{Использование динамической байесовой сети в задачах информационного поиска}.
  2009 г.

\bibitem{ndcg_book}
  Калевро Джарвелин, Яана Кикалайнен
  \emph{Накопленный прирост на основе оценки информационного посика. }.
  2002., стр 422–446.

\bibitem{cascade_click_book}
  Красвелл Н., Зойтер О.
  \emph{Эксперементальное сравнение моделей кликов, основанных на позициях.}.
  M: ACM, 2006. 87-94

\bibitem{norvig06}
  П. Норвиг, С. Рассел,
  \emph{Искусственный интеллект: современный подход}, 2-е изд.
  М.: Издательский дом ``Вильямс'',
  2006.

\bibitem{ot_1}
  Семенков, В.И.
  \emph{Охрана труда. Сборник нормативных правовых актов.}.
  В.И. Семенков, Липень Л.И. - Минск.: Дикта, 2009 - 784 с.

\bibitem{ot_2}
  [Электронный ресурс].
  \emph{Инструкции по охране труда.}.
  Электронные данные.  – Режим доступа : http://instruction-ot.by

\bibitem{decree432}
  Указ Президента РБ №432 от 31 августа 2009 года,
  \emph{О некоторых вопросах приобретения имущественных прав на результаты научно-технической деятельности и распоряжения этими правами}.
  http://president.gov.by/press76885.html

\bibitem{palitsyn06}
  Палицын В.А.,
  \emph{Технико-экономическое обоснование дипломных проектов. В 4-х частях. Часть 4: проекты программного обеспечения}.
  Мн.: БГУИР,
  2006.

\end{thebibliography}

\newpage