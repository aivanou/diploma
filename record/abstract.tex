\sectioncentered*{Реферат}
\thispagestyle{empty}

Дипломный проект выполнен на 6 листах формата А1 с пояснительной запиской на 80 страницах (с учетов трёх приложений). Пояснительная записка включает 6 глав, 15 рисунков, 22 литературных источника.

\emph{Ключевые слова}: поисковые системы, машинное обучение, логические деревья, байесовы сети, бустинг, обучение с учителем, ранжирование.


Целью данного дипломного проекта является исследдовательский проект, показывающий возможности улучшения релевантности поисковых технологий. Был рассмотрен стек алгоритмов, который использует как неявные отзывы от пользователей, так и векторы признаков документов. В конечном итоге, проект может быть внедрен в поисковые системы для улучшения ранжирования. Пояснительная записка состоит из следующих частей:

\emph{Введение}: описана актуальность проблемы в информационном поиске;

\emph{Глава 1}: описана предметная область, архитектура системы и меры, показывающие стойкость алгоритма; предлагается решение проблемы на основе алгоритмов машинного обучения;

\emph{Глава 2}: рассмотрен метод машинного обучения с учителем, основанный на бустинге для задачи ранжирования;

\emph{Глава 3}: описананы логические алгоритмы и деревья решения, которые используются бустингом в качестве слабых классификаторов

\emph{Глава 4}: описан метод нахождения релевантности документов из пользовательских сессий, основанный на динамической байесоовой сети

\emph{Глава 5}: рассмотрена безопасность инженеров-разработчиков в компании малого бизнеса Акавита

\emph{Глава 6}: приводится расчет сетевого графика и дается расчет экономического эффекта от использования программного средства;

\emph{Заключение}: содержит краткие выводы по дипломному проекту.

\newpage


